\documentclass[a4paper]{article}
\usepackage[left=3cm,right=3cm,top=2cm,bottom=2cm]{geometry} % page settings
\usepackage{enumerate}
\usepackage{hyperref}
\usepackage{graphicx}
\usepackage{amsfonts}
\usepackage{amsthm}
\usepackage{mathtools}
\usepackage[boxed]{algorithm2e}
\renewcommand{\algorithmcfname}{Algorytm}
\newcommand{\dime}[2]{${#1} \times {#2}$}
\SetKwInput{KwData}{\textbf{Dane}}
\SetKwInput{KwResult}{\textbf{Wynik}}
\SetAlgoSkip{smallskip}
\SetAlgoInsideSkip{medskip} 
\usepackage{titlesec}
\usepackage{polski}
\usepackage{tikz}
\usepackage[utf8]{inputenc}
\DeclarePairedDelimiter\ceil{\lceil}{\rceil}
\DeclarePairedDelimiter\floor{\lfloor}{\rfloor}
\DeclarePairedDelimiter\set{\lbrace}{\rbrace}
\newcommand{\rpm}{\raisebox{.2ex}{$\scriptstyle\pm$}}


\def\checkmark{\tikz\fill[scale=0.3](0,.35) -- (.25,0) -- (1,.7) -- (.25,.15) -- cycle;} 

\titlespacing*{\subsection}
{0ex}{10ex}{3ex}

\title{Gra z adwersarzem: zadanie 5.2}
\author{Kamil Matuszewski}
\date{\today}

\begin{document}

\maketitle
\setlength{\parindent}{0.5ex}
\setlength{\parskip}{1.5ex}
\newcommand{\R}{\mathbb{R}}
\newcommand{\N}{\mathbb{N}}

\subsection*{Treść}
Udowodnij, że $2n-1$ porównań trzeba wykonać, aby scalić dwa ciągi $n$ elementowe w modelu drzew decyzyjnych. Użyj gry z adwersarzem, który najpierw ogranicza przestrzeń danych do $2n$ tak, by każde porównanie eliminowało co najwyżej jeden zestaw.

\subsection*{Przestrzeń danych i jej ograniczenie}
Mamy więc dwa ciągi $n$ elementowe: $A=a_1,a_2,a_3,\dots ,a_n$ i $B=b_1, b_2, b_3, \dots , b_m$, i chcemy z nich utworzyć posortowany ciąg $X$, złożony ze wszystkich elementów z $A$ i wszystkich elementów z $B$ (więc jest długości $2n$). Adwersarz na początku przygotowuje sobie możliwe odpowiedzi - przestrzeń danych. Wszystkich możliwych ustawień ciągu złożonego z elementów z $A$ i z $B$ jest ${2n \choose n}$ - wybieramy miejsca dla elementów np. z ciągu $A$ z $2n$ dostępnych. Skoro $A$ i $B$ są posortowane (inaczej scalanie nie ma sensu) to wybranie miejsc dla któregoś z ciągów jednoznacznie daje nam ułożenie ciągu $X$. Adwersarz chce jednak jak najbardziej ograniczyć przestrzeń danych, by każde zapytanie eliminowało jak najmniej elementów a przestrzeń zdarzeń była jak największa. Innymi słowy chce, żeby gracz musiał wykonać jak najwięcej porównań by znaleźć odpowiedź.\\
Ograniczmy więc przestrzeń zdarzeń, tworząc ciągi, które wyglądają następująco: pierwszy ciąg to $X_0=a_1,b_1,a_2,b_2,\dots ,a_n,b_n=x_1,x_2,x_3,\dots,x_{2n}$ (oczywiście jest on posortowany rosnąco), a każdy kolejny $X_k$ dla $k>0$ tworzymy zamieniając ze sobą elementy $x_k$ i $x_{k+1}$. Takich zamian w ciągu o $2n$ elementach możemy wykonać $2n-1$. W sumie mamy więc $2n$ ciągów. To będzie nasza przestrzeń zdarzeń.\\
Dla jasności:
$$X_0=a_1,b_1,a_2,b_2,\dots , a_n,b_n$$
$$X_1=b_1,a_1,a_2,b_2,\dots , a_n,b_n$$
$$X_2=a_1,a_2,b_1,b_2,\dots , a_n,b_n$$
$$X_3=a_1,b_1,b_2,a_2,\dots , a_n,b_n$$
$$\vdots$$
$$X_{2n-1}=a_1,b_1,a_2,b_2,\dots , b_n,a_n$$
\clearpage
\subsection*{Zapytania gracza}
Po pierwsze załóżmy, że gracz nie zadaje głupich pytań, w stylu "Czy $a_i$ jest większe od $a_j$", gdyż ciągi $A$ i $B$ są posortowane, i takie pytania nic by nie wniosły (a gracz pragnie szybko wygrać).\\
Weźmy dowolne zapytanie "jakie jest $a_i$ w stosunku do $b_j$". Rozpatrzmy trzy przypadki:\\
\begin{itemize}
\item $i>j+1$. Adwersarz odpowiada wtedy $a_i$ jest większe od $b_j$. Zauważmy, że w ten sposób nie eliminuje żadnego ze swoich zestawów - kolejne zestawy to zamiana $b_i$ z $a_i$ oraz $b_i$ z $a_{i+1}$, a $a_i$ i $b_j$ różnią się co najmniej o 2.
\item $j>i$. Podobne do przypadku powyżej. Adwersarz odpowiada, że $a_i$ jest mniejsze od $b_j$. To pytanie również nie eliminuje żadnego z zestawów adwersarza, bowiem w kolejnych $x_k$ nie zamieniamy ze sobą elementów $a_i$ i $b_j$ dla $i<j$, a w $x_0$ dla $i<j$ zachodzi $a_i<b_j$.
\item $i=j$. Wtedy adwersarz odpowiada, że $a_i$ jest mniejsze od $b_j$. Zauważmy, że w ten sposób eliminuje przypadek, w którym $a_i$ jest większe od $b_i$ - jest to zamiana $x_{2i-1}=a_i$ i $x_{2i}=b_i$ w ciągu $X_0$, czyli ciąg $X_{2i-1}$.
\item $i=j+1$. Wtedy adwersarz odpowiada, że $a_i$ jest większe od $b_j$. Zauważmy, że w ten sposób eliminuje przypadek, w którym $a_i$ jest mniejsze od $b_{i-1}$ - jest to zamiana $x_{2i-2}=b_{i-1}$ i $x_{2i-1}=a_i$ w ciągu $X_0$, czyli ciąg $X_{2i-2}$.
\end{itemize} 
Pokazaliśmy więc, że dla dowolnego zapytania, usuniemy co najwyżej jeden zestaw.

\subsection*{Podsumowanie}
Gra z adwersarzem opiera się na prostych założeniach. Adwersarz jest troszkę niemiły, i na początku nie zna odpowiedzi na pytanie, choć twierdzi, że ją zna. Nie chce bowiem, by gracz zbyt szybko odgadł. Pragnie zmusić go do zadania jak największej liczy pytań. W tym celu najpierw tworzy sobie zestaw możliwych odpowiedzi tak, by było ich jak najwięcej, i każde pytanie eliminowało jak najmniej zestawów. W naszym przypadku adwersarz ograniczył sobie przestrzeń do $2n$, i każde jego pytanie eliminowało co najwyżej jeden z tych zestawów. Oznacza to, że potrzeba przynajmniej $2n-1$ pytań, by znaleźć odpowiedź w tym zestawie. Gdyby gracz miał strategię, która pozwala znaleźć odpowiedź wcześniej, w szczególności znalazłby ją wcześniej w tym zestawie, eliminując więcej niż jedną odpowiedź, co jak pokazaliśmy jest niemożliwe. To oznacza, że potrzeba $2n-1$ porównań, by w modelu drzew decyzyjnych scalić dwa ciągi.



 
\end{document}
