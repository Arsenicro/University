\documentclass[a4paper]{article}
\usepackage[left=3cm,right=3cm,top=2cm,bottom=2cm]{geometry} % page settings
\usepackage{enumerate}
\usepackage{hyperref}
\usepackage{graphicx}
\usepackage{amsfonts}
\usepackage{amsthm}
\usepackage{mathtools}
\usepackage{titlesec}
\usepackage{polski}
\usepackage{tikz}
\usepackage[utf8]{inputenc}
\DeclarePairedDelimiter\ceil{\lceil}{\rceil}
\DeclarePairedDelimiter\floor{\lfloor}{\rfloor}

\def\checkmark{\tikz\fill[scale=0.3](0,.35) -- (.25,0) -- (1,.7) -- (.25,.15) -- cycle;} 

\titlespacing*{\subsection}
{0ex}{10ex}{3ex}

\title{Lista 10}
\author{Kamil Matuszewski}
\date{\today}

\begin{document}

\maketitle
\setlength{\parindent}{0.5ex}
\setlength{\parskip}{1.5ex}
\newcommand{\R}{\mathbb{R}}

\begin{center}
\begin{tabular}{|c *{7}{|c} |c|}\hline
1 & 2 & 3 & 4 & 5 & 6 & 7\\
\hline 
\checkmark & \checkmark & \checkmark &  & \checkmark &  &  \\
\hline
\end{tabular}\\
\end{center}

\subsection*{Zadanie 1}
Musimy sprawdzić trzy warunki:\\

\begin{itemize}
  \item $|| f || = 0 \Rightarrow f=0$
  $$|| f || = \sqrt{\sum\limits_{k=0}^{N}p(x_k)f(x_k)^2} $$
\begin{center}
	Wiemy, że $p(x_k) > 0$ $\forall x_k \in X$, oraz, że $f(x_k)^2 \geq 0$ i $f(x_k)^2=0 \Rightarrow f(x_k) = 0$ (bo to kwadrat).\\
	Suma jest nieujemna, więc jest zerem tylko wtedy, kiedy każdy jej składnik jest zerem. Skoro tak to $f=0$.
\end{center}
	\item $||\alpha x|| = | \alpha |\cdot ||x|| $
	 $$|| \alpha f || = \sqrt{\sum\limits_{k=0}^{N}p(x_k)(\alpha f(x_k))^2} $$
	 $$|| \alpha f || = \sqrt{\sum\limits_{k=0}^{N}p(x_k)\alpha^2 f(x_k)^2} $$
	 $$|| \alpha f || = \sqrt{\alpha^2 \sum\limits_{k=0}^{N}p(x_k) f(x_k)^2} $$
     $$|| \alpha f || = | \alpha | \cdot \sqrt{\sum\limits_{k=0}^{N}p(x_k) f(x_k)^2} $$
     $$|| \alpha f || = | \alpha | \cdot || f || $$

	\item $ ||x+y||\leq ||x|| + ||y|| $
	$$\sqrt{\sum\limits_{k=0}^{N}p(w_k)(x_k+y_k)^2} \leq \sqrt{\sum\limits_{k=0}^{N}p(w_k)x_k^2} + \sqrt{\sum\limits_{k=0}^{N}p(w_k)y_k^2}$$
	$$\sqrt{\sum\limits_{k=0}^{N}p(w_k)(x_k^2+2x_ky_k+y_k^2)} \leq \sqrt{\sum\limits_{k=0}^{N}p(w_k)x_k^2} + \sqrt{\sum\limits_{k=0}^{N}p(w_k)y_k^2}$$
	$$ \sqrt{\sum\limits_{k=0}^{N}p(w_k)x_k^2 + 2\sum\limits_{k=0}^{N}p(w_k)x_ky_k + \sum\limits_{k=0}^{N}p(w_k)y_k^2} \leq \sqrt{\sum\limits_{k=0}^{N}p(w_k)x_k^2} + \sqrt{\sum\limits_{k=0}^{N}p(w_k)y_k^2}$$ 
	Obie strony są nieujemne więc można podnieść obie strony do kwadratu:
	$$ \sum\limits_{k=0}^{N}p(w_k)x_k^2 + 2\sum\limits_{k=0}^{N}p(w_k)x_ky_k + \sum\limits_{k=0}^{N}p(w_k)y_k^2 \leq \sum\limits_{k=0}^{N}p(w_k)x_k^2 + \sum\limits_{k=0}^{N}p(w_k)y_k^2 + 2\sqrt{\sum\limits_{k=0}^{N}p(w_k)x_k^2}\cdot \sqrt{\sum\limits_{k=0}^{N}p(w_k)y_k^2}$$ 
	$$ \sum\limits_{k=0}^{N}p(w_k)x_ky_k \leq \sqrt{\sum\limits_{k=0}^{N}p(w_k)x_k^2 \cdot \sum\limits_{k=0}^{N}p(w_k)y_k^2}$$ 
Musimy to sprawdzić. W tym celu pokażę, że:
$$\left( \sum\limits_{k=0}^{N}p(w_k)x_ky_k \right)^2 \leq  \sum\limits_{k=0}^{N}p(w_k)x_k^2 \cdot \sum\limits_{k=0}^{N}p(w_k)y_k^2$$
$$\sum\limits_{k=0}^{N}p(w_k)x_k^2 \cdot \sum\limits_{k=0}^{N}p(w_k)y_k^2 - \left( \sum\limits_{k=0}^{N}p(w_k)x_ky_k \right)^2 \geq 0$$
$$\sum\limits_{0\leq i,j \leq N}p(w_i)p(w_j) x_i^2 y_j^2 - \sum\limits_{0 \leq i \leq N} p(w_i)^2 x_i^2 y_i^2 - \sum\limits_{0\leq i < j \leq N} 2p(w_i)p(w_j)x_ix_jy_iy_j =$$ 
$$= \sum\limits_{0\leq i\neq j \leq N}p(w_i)p(w_j) x_i^2 y_j^2 - \sum\limits_{0\leq i < j \leq N} 2p(w_i)p(w_j)x_ix_jy_iy_j =$$ 
$$=\sum\limits_{0 \leq i < j \leq N}p(w_i)p(w_j)x_i^2y_j^2 + p(w_j)p(w_i)x_j^2y_i^2 - \sum\limits_{0\leq i < j \leq N} 2p(w_i)p(w_j)x_ix_jy_iy_j =$$
$$=\sum\limits_{0 \leq i < j \leq N}p(w_i)p(w_j)(x_i^2 y_j^2 + x_j^2 y_i^2 - 2x_i x_j y_i y_j) =\sum\limits_{0 \leq i < j \leq N}p(w_i)p(w_j)(x_iy_j - x_jy_i)^2$$
Wiemy, że $p(w)>0$ $\forall w \in X$. Skoro tak, to każdy składnik tej sumy jest $\geq$ 0, czyli:
$\sum\limits_{0 \leq i < j \leq N}p(w_i)p(w_j)(x_iy_j - x_jy_i)^2 \geq 0$ więc:
$$\sum\limits_{k=0}^{N}p(w_k)x_k^2 \cdot \sum\limits_{k=0}^{N}p(w_k)y_k^2 \geq \left( \sum\limits_{k=0}^{N}p(w_k)x_ky_k \right)^2$$
$$\sqrt{\sum\limits_{k=0}^{N}p(w_k)x_k^2 \cdot \sum\limits_{k=0}^{N}p(w_k)y_k^2} \geq \sqrt{\left( \sum\limits_{k=0}^{N}p(w_k)x_ky_k \right)^2}\geq \sum\limits_{k=0}^{N}p(w_k)x_ky_k$$
A to jest to co chcieliśmy pokazać.
\end{itemize}

\subsection*{Zadanie 2}
Mamy model:
$$y(x)=ax+2015$$
$y \in Y$ : $|| f-y ||_2^2 = \sum\limits_{k=0}^n (f(x_k) - y(x_k))^2 = \sum\limits_{k=0}^n (f(x_k) - ax_k - 2015)^2 = \sum\limits_{k=0}^n (f(x_k) - 2015 - ax_k)^2 = E(a)$
$$y^*(x)=a^* : ||f-y^*||_2 = \min\limits_{y \in Y}||f-y||_2=\min\limits_{a \in \R}\sqrt{E(a)}$$
Pierwiastek nie wpływa na monotoniczność, stąd wystarczy policzyć pochodną funkcji $E(a)$
$$E'(a) =  \left( \sum\limits_{k=0}^n (f(x_k) - 2015 - ax_k)^2 \right)' = \left( \sum\limits_{k=0}^n (f(x_k) - 2015)^2 - 2ax_k(f(x_k) - 2015) + (ax_k)^2 \right)' = $$ $$ = -2 \sum\limits_{k=0}^n x_k(f(x_k) - 2015 - ax_k) $$
$$-2 \sum\limits_{k=0}^n x_k(f(x_k) - 2015 - ax_k) = 0$$
$$ \sum\limits_{k=0}^n f(x_k)x_k - 2015x_k -  \sum\limits_{k=0}^n ax_k^2 = 0 $$
$$ \sum\limits_{k=0}^n f(x_k)x_k - 2015x_k =  \sum\limits_{k=0}^n ax_k^2 $$
$$ \sum\limits_{k=0}^n f(x_k)x_k - 2015x_k =  a\sum\limits_{k=0}^n x_k^2 $$
$$ \frac{\sum\limits_{k=0}^n x_k(f(x_k) - 2015)}{\sum\limits_{k=0}^n x_k^2} =  a$$


\subsection*{Zadanie 3}
Wystarczy znaleźć ekstremum funkcji:
$$\sum\limits_{k=0}^r \frac{1}{\log{(1+x_k^2)}}(y_k-a\sin{x_k})^2 $$
Łatwo sprawdzić, że pochodną jest:\\
$$-2 \sum\limits_{k=0}^{r} \frac{\sin{x_k}}{\log{(1+x_k^2)}}\cdot (y_k-a\sin{x_k})$$
Teraz przyrównujemy do 0:
$$-2 \sum\limits_{k=0}^{r} \frac{\sin{x_k}y_k}{\log{(1+x_k^2)}} - \frac{a\sin^2{x_k}}{\log{(1+x_k^2)}}= 0$$
$$ \sum\limits_{k=0}^{r} \frac{\sin{x_k}\cdot y_k}{\log{(1+x_k^2)}} - \sum\limits_{k=0}^{r} \frac{a\sin^2{x_k}}{\log{(1+x_k^2)}} = 0$$
$$ \sum\limits_{k=0}^{r} \frac{\sin{x_k}\cdot y_k}{\log{(1+x_k^2)}} = a\cdot \sum\limits_{k=0}^{r} \frac{\sin^2{x_k}}{\log{(1+x_k^2)}} $$
$$ a = \frac{\sum\limits_{k=0}^{r} \frac{\sin{x_k}\cdot y_k}{\log{(1+x_k^2)}}}{\sum\limits_{k=0}^{r} \frac{\sin^2{x_k}}{\log{(1+x_k^2)}}}$$

\subsection*{Zadanie 4}
Wystarczy zaproksymować funkcję odwrotną (trywialne)

\subsection*{Zadanie 5}
Na wykładzie, pokazaliśmy, że dla modelu $$Y=ax+b$$ Rozwiązaniem jest $$a=\frac{(n+1)S_4 - S_1S_3}{(n+1)S_2 - S_1^2} $$ $$b=\frac{S_2S_3-S_1S_4}{(n+1)S_2 - S_1^2} $$
Gdzie:
$$\begin{matrix*}[l]
S_i = \sum\limits_{k=0}^n x_k^i &&&& i=1,2\\
S_3 = \sum\limits_{k=0}^n f(x_k)\\
S_4 = \sum\limits_{k=0}^n x_k f(x_k)
\end{matrix*}$$
Co można pokazać poprzez pochodną dwóch zmiennych i regresję liniową.
$$
\begin{matrix*}[l]
n=7\\
S_1=10+20+30+40+80+90+95=365\\
S_2=100+400+900+1600+6400+8100+9025=26525\\
S_3=514,5\\
S_4=671+1328+1968+2584+4944+5490+5700=22685\\
\end{matrix*}
$$
Stąd:\\
$$
\begin{matrix*}[l]
a=-0,08\\
b=67,96\\
\end{matrix*}
$$
Więc:\\
$$ 
S=-0,08\cdot T + 67,96
$$
Co jest zgodne z podaną tabelką(z jakąś dokładnością).

\subsection*{Zadanie 6}


\subsection*{Zadanie 7}
Liczymy pochodną wyrażenia, i porządkujemy.
Oznaczmy sobie:
$$ $$

\end{document}
