\documentclass[a4paper]{article}
\usepackage[left=3cm,right=3cm,top=2cm,bottom=2cm]{geometry} % page settings
\usepackage{enumerate}
\usepackage{hyperref}
\usepackage{graphicx}
\usepackage{amsfonts}
\usepackage{amsthm}
\usepackage{mathtools}
\usepackage{titlesec}
\usepackage{polski}
\usepackage{tikz}
\usepackage[utf8]{inputenc}
\DeclarePairedDelimiter\ceil{\lceil}{\rceil}
\DeclarePairedDelimiter\floor{\lfloor}{\rfloor}
\DeclarePairedDelimiter\set{\lbrace}{\rbrace}


\def\checkmark{\tikz\fill[scale=0.3](0,.35) -- (.25,0) -- (1,.7) -- (.25,.15) -- cycle;} 

\titlespacing*{\subsection}
{0ex}{10ex}{3ex}

\title{Lista 1}
\author{Kamil Matuszewski}
\date{\today}

\begin{document}

\maketitle
\setlength{\parindent}{0.5ex}
\setlength{\parskip}{1.5ex}
\newcommand{\R}{\mathbb{R}}
\newcommand{\N}{\mathbb{N}}


\begin{center}
\begin{tabular}{|c *{8}{|c} |c|}\hline
1 & 2 & 3 & 4 & 5 & 6 & 7 & 8 & 9 & 10\\
\hline 
\checkmark &\checkmark &\checkmark &\checkmark &\checkmark &\checkmark &\checkmark &\checkmark & &\\
\hline
\end{tabular}
\end{center}

\subsection*{Zadanie 1}
A oraz B są zdarzeniami takimi, że: $P(A \cap B)$ = $\frac{1}{4}$, $P(A^C)$ = $\frac{1}{3}$, $P(B)$ = $\frac{1}{2}$. Znaleźć $P(A \cup B)$.

Prawdopodobieństwo musi się sumować do 1, więc skoro $P(A^C)=\frac{1}{3}$ to $P(A)=\frac{2}{3}$. Prawdopodobieństwo $P(A\cup B)$ to $P(A)+P(B)$. Ale zdarzenia A i B mogą się pokrywać, więc odejmujemy $P(A\cap B)$, i otrzymujemy: $P(A\cup B)=P(A)+P(B)-P(A\cap B) = \frac{2}{3} + \frac{1}{2}-\frac{1}{4}=\frac{3}{4}$

\subsection*{Zadanie 2}
Dane są niezależne zmienne losowe $X$ o rozkładzie $B(n_1, p)$ oraz $Y$ o rozkładzie $B(n_2, p)$. Wykazać, że zmienna $Z = X + Y$ ma rozkład $B(n_1 + n_2, p)$.

$$P(Z=k)=\sum\limits_{i=1}^kP(X=i,Y=k-i) =\sum\limits_{i=1}^kP(X=i)P(Y=k-i)=\sum_{i=1}^k {{n_1} \choose i} p^i(1-p)^{n_1 - i} {{n_2} \choose {k-i}} p^{k-i}(1-p)^{n_2-k+i}=$$ $$= \sum_{i=1}^k {{n_1} \choose i}{{n_2} \choose {k-i}}p^{k}(1-p)^{n_1+n_2-k}=p^k(1-p)^{n_1 + n_2 -k}\sum_{i=1}^k {n_1 \choose i} {n_2 \choose k-i}\stackrel{*}{=}p^k(1-p)^{n_1 + n_2 -k}{n_1+n_2 \choose k}=$$ $$=B(n_1 + n_2, p).$$

* - tożsamość Cauchy’ego, pokazywana na MDM.

Mamy więc rozkład $B(n_1 + n_2, p)$ zmiennej $Z=X+Y$
\clearpage
\subsection*{Zadanie 3}
Dane są niezależne zmienne losowe X,Y o rozkładach Poissona $\lambda_1$ i $\lambda_2$.Wykazać, że zmienna $Z=X+Y$ ma rozkład Poissona z parametrami $\lambda_1+\lambda_2$

$$P(Z=k)=\sum_{i=0}^k P(X=i,Y=k-i)= \sum_{i=0}^k P(X=i)\cdot P(Y=k-i)=\sum_{i=0}^k e^{-\lambda_1} \frac {\lambda_1^i}{i!} \cdot e^{-\lambda_2} \frac{\lambda_2^{k-i}}{(k-i)!}=$$ $$= e^{-(\lambda_1+\lambda_2)} \sum_{i=0}^k \frac{\lambda_1^i}{i!} \cdot \frac{\lambda_2^{k-i}}{(k-i)!}=e^{-(\lambda_1+\lambda_2)} \sum_{i=0}^k \frac{1}{k!} \cdot \frac{k!}{i!\cdot(k-i)!} \lambda_1^i\lambda_2^{k-i} = e^{-(\lambda_1+\lambda_2)} \frac{1}{k!} \sum_{i=0}^k {k \choose i}\lambda_1^i\lambda_2^{k-i}=$$ $$= e^{-(\lambda_1+\lambda_2)} \frac{(\lambda_1+\lambda_2)^k}{k!} $$

Co daje nam to co chcieliśmy.

\subsection*{Zadanie 4}
Prawdopodobieństwo sukcesu w jednej próbie jest równe p. Wykonujemy (niezależne) próby do otrzymania sukcesu. Wyznaczyć rozkład i wartość oczekiwaną X.

Nie wiem czy dobrze kombinuję:\\
Wykonujemy próbę do uzyskania sukcesu. Skoro tak, to przegrywamy $n-1$ razy z prawdopodobieństwem $(1-p)$, a następnie wygrywamy raz z prawdopodobieństwem $p$, stąd nasz rozkład to $p(1-p)^{n-1}$.\\
Teraz, wartość oczekiwana to:\\
$$E(X)=\sum\limits_{k=0}^\infty k P(X=k)=\sum\limits_{k=0}^\infty k \cdot p(1-p)^{k-1}$$

\subsection*{Zadanie 5}
Prawdopodobieństwo sukcesu w jednej próbie jest równe p. Wykonujemy (niezależne) próby do otrzymania 2 sukcesów. Wyznaczyć rozkład i wartość oczekiwaną X.

Podobnie jak 4:\\
Wykonujemy $n-1$ prób, i wśród nich musi być jedna wygrana. Stąd, z rozkładu Bernoulliego, prawdopodobieństwo to:\\
$$P=(n-1)p(1-p)^{n-2}$$
Na sam koniec musimy raz wygrać, z prawdopodobieństwem $p$, stąd ostateczny wzór to:\\
$$P=(n-1)p(1-p)^{n-2}\cdot p=(n-1)p^2(1-p)^{n-2}$$\\
Teraz, wartość oczekiwana:\\
$$E(X)=\sum\limits_{k=0}^\infty k P(X=k)=\sum\limits_{k=0}^\infty k \cdot (k-1)p^2(1-p)^{k-2}$$

\subsection*{Zadanie 6}
Ok, pobawmy się w wypełnianie tabelki. Jeśli $X=\set{0,1,2,4}$, wtedy łatwo zauważyć, że prawdopodobieństwa wylosowania danych kolorów to zawsze $\frac{1}{4}$, bo w każdym kolorze jest po 6 kart, i są 24 karty. Stąd prawdopodobieństwa, to kolejno $P(X)=\set{\frac{1}{4},\frac{1}{4},\frac{1}{4},\frac{1}{4}}$\\
Trochę inaczej sprawa się ma z $Y=\set{0,2,4,5}$. Zastanówmy się najpierw nad asem królem i damą. Są po 4 figury w talii, i mamy 24 karty, stąd prawdopodobieństwo wyciągnięcia danej karty wynosi $\frac{4}{24}=\frac{1}{6}$. Co do wylosowania jakiejś pozostałej, to skoro mamy 4 asy, 4 damy i 4 króle, to zostaje nam 12 kart, więc prawdopodobieństwo to $\frac{12}{24}=\frac{1}{2}$. Stąd prawdopodobieństwa to kolejno $Y=\set{\frac{1}{2},\frac{1}{6},\frac{1}{6},\frac{1}{6}}$. Zauważmy, że zrobiliśmy właśnie rozkład brzegowy! Super, pół zadania z głowy. Teraz tylko tabelka. Wykonujemy ją tak, że mnożymy prawdopodobieństwo wylosowania X z prawdopodobieństwem wylosowania Y. Stąd otrzymujemy tabelkę:

\begin{tabular}{c *{4}{|c}}
X$\backslash$Y & 0 & 2 & 4 & 5 \\\hline
0 & $\frac{1}{8}$ $(0)$ & $\frac{1}{24}$ $(2)$ & $\frac{1}{24}$ $(4)$ & $\frac{1}{24}$ $(5)$\\\hline
1 & $\frac{1}{8}$ $(1)$ & $\frac{1}{24}$ $(3)$ & $\frac{1}{24}$ $(5)$ & $\frac{1}{24}$ $(6)$\\\hline
2 & $\frac{1}{8}$ $(2)$ & $\frac{1}{24}$ $(4)$ & $\frac{1}{24}$ $(6)$ & $\frac{1}{24}$ $(7)$\\\hline
4 &	$\frac{1}{8}$ $(4)$ & $\frac{1}{24}$ $(6)$ & $\frac{1}{24}$ $(8)$& $\frac{1}{24}$ $(9)$\\
\end{tabular}\\
(Aktualnie nie przejmujcie się tymi liczbami w nawiasach, przydadzą się do zad 8).
Wszystko zgadza się z intuicją(w szczególności suma poszczególnych kolumn daje nam rozkład brzegowy, a cała tabelka sumuje się do 1) więc jest szansa, że nic nie pojebałem.

\subsection*{Zadanie 7}
Mamy sprawdzić, czy zmienne X i Y są niezależne. Ale to wyszło już przy tworzeniu tabelki - szansa, że wylosujemy np kiera nie oznacza, że mamy większe/mniejsze szanse na wylosowanie damy. Więc mamy odpowiedź.

\subsection*{Zadanie 8}
Rozkład prawdopodobieństwa zmiennej $Z=X+Y$. Zapowiedziane w zadaniu 6 liczby w nawiasach to właśnie nasze Z. Teraz musimy tylko zsumować prawdopodobieństwa.

\begin{tabular}{c *{11}{c|} c}
Z && 0 & 1 & 2 & 3 & 4 & 5 & 6 & 7 & 8 & 9\\\hline
&&  $\frac{1}{8}$ & $\frac{1}{8}$ & $\frac{1}{6}$ & $\frac{1}{24}$ & $\frac{5}{24}$ & $\frac{1}{12}$ & $\frac{3}{24}$ & $\frac{1}{24}$ & $\frac{1}{24}$ & $\frac{1}{24}$\\
\end{tabular}


\end{document}
