\documentclass[a4paper]{article}
\usepackage[left=3cm,right=3cm,top=2cm,bottom=2cm]{geometry} % page settings
\usepackage{enumerate}
\usepackage{hyperref}
\usepackage{graphicx}
\usepackage{amsfonts}
\usepackage{amsthm}
\usepackage{mathtools}
\usepackage[boxed]{algorithm2e}
\renewcommand{\algorithmcfname}{Algorytm}
\newcommand{\dime}[2]{${#1} \times {#2}$}
\SetKwInput{KwData}{\textbf{Dane}}
\SetKwInput{KwResult}{\textbf{Wynik}}
\SetAlgoSkip{smallskip}
\SetAlgoInsideSkip{medskip} 
\usepackage{titlesec}
\usepackage{polski}
\usepackage{tikz}
\usepackage[utf8]{inputenc}
\DeclarePairedDelimiter\ceil{\lceil}{\rceil}
\DeclarePairedDelimiter\floor{\lfloor}{\rfloor}
\DeclarePairedDelimiter\set{\lbrace}{\rbrace}
\newcommand{\rpm}{\raisebox{.2ex}{$\scriptstyle\pm$}}


\def\checkmark{\tikz\fill[scale=0.3](0,.35) -- (.25,0) -- (1,.7) -- (.25,.15) -- cycle;} 

\titlespacing*{\subsection}
{0ex}{10ex}{3ex}

\title{Lista 6}
\author{Kamil Matuszewski}
\date{\today}

\begin{document}

\maketitle
\setlength{\parindent}{0.5ex}
\setlength{\parskip}{1.5ex}
\newcommand{\R}{\mathbb{R}}
\newcommand{\N}{\mathbb{N}}


\begin{center}
\begin{tabular}{|c *{8}{|c} |c|}\hline
1 & 2 & 3 & 4 & 5 & 6 & 7 & 8 & 9 & 10\\
\hline 
 & & & & & & & & &\\
\hline
\end{tabular}\\
\end{center}

\subsection*{Zadanie 1}
Znajdź estymator $Geom(p)$ parametru $p$.

$$f(x;p)=(1-p)^{x-1}p$$
$$f(x_1,x_2,\dots ,x_n, p)=p^n(1-p)^{\sum_1^n x_i - n} = L(p)$$
$$\log{L(p)} = n\log{p} + \left( \sum\limits_1^n x_i - n\right) \log{(1-p)} $$
$$ \frac{d[\log{L(p)}]}{p} = \frac{n}{p} + \frac{\sum_1^n x_i - n}{1-p} = 0 \Rightarrow p=\frac{n}{\sum_i^n x_i}$$

\subsection*{Zadanie 2,3}
Znajdź estymator $Pareto(x;a,k)$ zmiennej $a$ oraz zmiennej $k$.

$$f(x;a,k)=\frac{ka^k}{x^{k+1}}$$
$$L(a,k)=f(x_1,\dots ,x_n;a,k)=\prod\limits_{i=1}^n \frac{ka^k}{x_i^{k+1}}$$
$$\log{L(k,a)}=\sum\limits_{i=1}^n \log{\frac{k a^k}{x_i^{k + 1}}}=n\log{k} + k n \log{a} - (k+1)  \sum\limits_{i=1}^n \log{x_i}$$


Zmienna a:\\
Zastanówmy się najpierw, z jakiego przedziału jest $a$ a raczej jakie maksymalne $a$ możemy podać. Maksymalnym $a$ jakie możemy podać jest $min(x_k)$, bo $x\in (a,\infty)$.\\
Mamy:
$$\log{L(k,a)}= n\log{k} + k n \log{a} - (k+1)  \sum\limits_{i=1}^n \log{x_i}$$
Chcemy to z maksymalizować za pomocą $a$, kiedy $k,n,x_i$ są dane. Żeby to zrobić, możemy tylko zmaksymalizować $\log{a}$. W takmi razie musimy wziąć największą możliwą wartość $a$ którą jest $min(x_i)$, co jest naszym estymatorem.

Zmienna k:
$$\frac{d[\log{L(k,a)]}}{k} = \frac{n}{k} + n\log{a} - \sum\limits_{i=1}^n \log{x_i} = 0 $$
$$\frac{n}{k} -  \sum\limits_{i=1}^n \log{\frac{x_i}{a}} \Rightarrow k=\frac{n}{ \sum\limits_{i=1}^n \log{\frac{x_i}{a}}}$$


\subsection*{Zadanie 4}
Znajdź estymator rozkładu wykładniczego $f(x;\lambda)$ parametru $\lambda$.

$$f(x;\lambda)=\lambda e^{-\lambda x}$$
$$f(x_1,\dots ,x_n;\lambda)=\lambda^n e^{- \lambda \sum x}=L(\lambda)$$
$$\log{L(\lambda)} = n log(\lambda) - \lambda \sum x$$
$$\frac{d[\log{L(\lambda)}]}{\lambda} = \frac{n}{\lambda} - \sum x  = 0 \Rightarrow \lambda = \frac{n}{\sum x} $$

\subsection*{Zadanie 5}
Znajdź estymator rozkładu Weibulla $f(x;k,\lambda)$ parametru $\lambda$.

$$f(x;k,\lambda) = \frac{k}{\lambda} \left( \frac{x}{\lambda} \right)^{k-1}e^{-\left( \frac{x}{\lambda}\right )^k} $$
$$f(x_1,x_2,\dots ,x_n;k,\lambda) = \frac{k^n}{\lambda^{nk}}e^{-\sum_1^n \left( \frac{x_i}{\lambda} \right)^k  } \prod\limits_{i=1}^n x_i^{k-1} = L(\lambda)  $$
$$\log{L(\lambda)} = n\log{k} - nk\log{\lambda} - \sum_{i=1}^n \left( \frac{x_i}{\lambda} \right)^k + (k-1)\sum_{i=1}^n x_i  $$
$$\frac{\log{L(\lambda)}}{\lambda} = \frac{-nk}{\lambda} + k \sum_{i=1}^n \frac{x_i^k}{\lambda^{k+1}}=0$$
$$\frac{-nk}{\lambda} + k \sum_{i=1}^n \frac{x_i^k}{\lambda^{k+1}}=0 $$
$$\frac{-nk}{\lambda} + \frac{k}{\lambda} \sum_{i=1}^n \frac{x_i^k}{\lambda^{k}}=0 \ \ \ \ \ \backslash : \frac{-nk}{\lambda} $$
$$-1 + \frac{1}{n} \sum_{i=1}^n \frac{x_i^k}{\lambda^{k}}=0$$
$$\frac{1}{n} \frac{1}{\lambda^{k}} \sum_{i=1}^n x_i^k=1$$
$$\frac{1}{n} \sum_{i=1}^n x_i^k=\lambda^k \Rightarrow \lambda = \left( \frac{\sum_1^n x_i^k}{n} \right)^{-k}$$

\subsection*{Zadanie 6}
Mamy $X\sim \chi^2(n)$ i $Y\sim \chi^2(k)$ - niezależne. Znajdź gęstość $(X,Y)$

$$f_X(x)=\frac{\left( \frac{1}{2} \right)^\frac{n}{2}}{\Gamma(\frac{n}{2})} x^{\frac{n}{2}-1}e^{\frac{-x}{2}} $$
$$f_Y(y)=\frac{\left( \frac{1}{2} \right)^\frac{k}{2}}{\Gamma(\frac{k}{2})} y^{\frac{k}{2}-1}e^{\frac{-y}{2}} $$
$$f_{(X,Y)}(x,y) = \frac{\left( \frac{1}{2} \right)^\frac{n}{2}}{\Gamma(\frac{n}{2})} x^{\frac{n}{2}-1}e^{\frac{-x}{2}} \cdot  \frac{\left( \frac{1}{2} \right)^\frac{k}{2}}{\Gamma(\frac{k}{2})} y^{\frac{k}{2}-1}e^{\frac{-y}{2}}$$

\end{document}
