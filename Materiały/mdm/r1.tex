\documentclass[a4paper]{article}
\usepackage[left=3cm,right=3cm,top=2cm,bottom=2cm]{geometry} % page settings
\usepackage{enumerate}
\usepackage{hyperref}
\usepackage{graphicx}
\usepackage{amsfonts}
\usepackage{amsthm}
\usepackage{mathtools}
\usepackage{titlesec}
\usepackage{polski}
\usepackage{tikz}
\usepackage[utf8]{inputenc}
\DeclarePairedDelimiter\ceil{\lceil}{\rceil}
\DeclarePairedDelimiter\floor{\lfloor}{\rfloor}

\def\checkmark{\tikz\fill[scale=0.3](0,.35) -- (.25,0) -- (1,.7) -- (.25,.15) -- cycle;} 

\titlespacing*{\subsection}
{0ex}{10ex}{3ex}

\title{Lista 1}
\author{Kamil Matuszewski}
\date{8 października 2015}

\begin{document}

\maketitle
\setlength{\parindent}{0.5ex}
\setlength{\parskip}{1.5ex}

\begin{center}
\begin{tabular}{|c *{16}{|c} |c|}\hline
1 & 2 & 3 & 4 & 5 & 6 & 7 & 8 & 9 & 10 & 11 & 12 & 13 & 14 & 15\\
\hline 
 &\checkmark &  & & \checkmark & \checkmark & \checkmark & \checkmark &  & & & & \checkmark &  & \\
\hline
\end{tabular}\\
\end{center}

\subsection*{Zadanie 2}
Mamy dane funkcje:

$$\log{n} $$
$$(\log{n})^n=2^{n\cdot \log{\log{n}}} $$
$$n^{\log{n}}=2^{\log{n}\log{n}}=2^{log^2{n}} $$
$$\log{n^n} = n\log{n}$$
$$3^{\log{n}} = 2^{\log{3}\log{n}} = 2^{\log{n} \log{3}}=n^{\log{3}}$$
$$n$$
$$n^2$$
$$2^{\sqrt{n}} $$
$$(1,01)^n = 2^{\log{1,01}n}$$
$$(0,99)^n \underset{n\rightarrow \infty}{=} 0  $$
$$(n+\frac{1}{n})^n \underset{n\rightarrow \infty}{=} e  $$

Korzystając z wiedzy z wykładu oraz z powyższych równań możemy uporządkować funkcje:

$$(0,99)^n, (n+\frac{1}{n})^n, \log{n}, n, \log{n^n}, 3^{\log{n}}, n^2, n^{\log{n}}, 2^{\sqrt{n}}, (1,01)^n, \log{n}^n$$


\subsection*{Zadanie 5}
Wykażemy, że:

\begin{enumerate}[(a)]
\item $f=o(g) \Rightarrow f=O(g) $
\begin{proof}
$$f(n)=o(g(n)) \overset{def}{\Leftrightarrow} \lim_{n\rightarrow \infty} \frac{f(n)}{g(n)}=0 \overset{def. Cauchy'ego}{\Rightarrow} \forall_{\epsilon>0} \exists_{n_0 > 0} \forall_{n > n_0} |\frac{f(n)}{g(n)} - 0| \leq \epsilon \Leftrightarrow$$
$$\Leftrightarrow \forall_{\epsilon>0} \exists_{n_0 > 0} \forall_{n > n_0} |\frac{f(n)}{g(n)}| \leq \epsilon \Leftrightarrow \forall_{\epsilon>0} \exists_{n_0 > 0} \forall_{n > n_0} |f(n)| \leq \epsilon \cdot |g(n)| \overset{def}{\Leftrightarrow} f(n)=O(g(n)) $$
\end{proof}
\item $f\sim g \Rightarrow f=\Theta(g)$

\begin{proof}
$$f\sim g \overset{def}{\Leftrightarrow} \lim_{n\rightarrow \infty} \frac{f(n)}{g(n)}=1 \overset{def. Cauchy'ego}{\Rightarrow} \forall_{\epsilon>0} \exists_{n_0 > 0} \forall_{n > n_0} |\frac{f(n)}{g(n)} - 1| \leq \epsilon$$
To oznacza, że:
$$-\epsilon \leq \frac{f(n)}{g(n)} - 1 \leq \epsilon$$
$$1-\epsilon \leq \frac{f(n)}{g(n)}\leq 1+\epsilon$$
$$(1-\epsilon)g(n) \leq f(n)\leq (1+\epsilon)g(n)$$
Zachodzi dla każdego $\epsilon>0$, więc w szczególności $\exists_{\epsilon} (1-\epsilon)>0 \wedge (1+\epsilon)>0$.\\
Niech $(1-\epsilon)=c$ i $(1+\epsilon)=d$. Z powyższego wiemy, że:
$$\exists_{c>0,d>0,n_0>0} \forall_{n>n_0} c\cdot g(n) \leq f(n) \leq d \cdot g(n) \overset{def}{\Leftrightarrow} f(n)=\Theta(g)$$ 
\end{proof}

\item $f=O(g) \Leftrightarrow g=\Omega(f)$
\begin{proof}
$$f=O(g) \overset{def}{\Leftrightarrow} \exists_{c>0, n_0} \forall_{n>n_0} |f(n)|\leq c|g(n)| \Leftrightarrow \exists_{c>0} g(n)\geq \frac{1}{c} |f(n)| \Leftrightarrow g=\Omega(f)$$
\end{proof}

\item $f=O(g) \wedge g=O(f) \Leftrightarrow g=\Omega(f)$
\begin{proof}
$$f=O(g) \overset{def}{\Leftrightarrow} \exists_{c>0, n_0} \forall_{n>n_0} |f(n)|\leq c|g(n)| $$
$$g=O(f) \overset{def}{\Leftrightarrow} \exists_{d>0, n_0} \forall_{n>n_0} |g(n)|\leq d|f(n)| $$
Więc:
$$\exists_{c>0, d>0, n_0} \forall_{n>n_0} \frac{1}{c}|f(n)|\leq g(n) \leq d|f(n)|\overset{def}{\Leftrightarrow} g=\Theta(f)$$
\end{proof}

\end{enumerate}

Przechodnie: Wszystkie\\
Symetryczne: $\sim, \Theta$
Wynika wprost z definicji.

\clearpage
\subsection*{Zadanie 6}
Pokaż, że
$$e^{1/n} = 1+\frac{1}{n} + O\left(\frac{1}{n^2}\right)$$

\begin{proof}
$$e^{x}=\sum\limits_{i=0}^{\infty} \frac{x^i}{i!} $$
$$e^{1/n}=\sum\limits_{i=0}^{\infty} \frac{(\frac{1}{n})^i}{i!} =\sum\limits_{i=0}^{\infty} \frac{1}{n^i i!} = 1+ \frac{1}{n} + \sum\limits_{i=2}^{\infty} \frac{1}{n^i i!}$$
$$\sum\limits_{i=2}^{\infty} \frac{1}{n^i i!} \leq \sum\limits_{i=2}^{\infty} \frac{1}{n^2 2^i} = \frac{1}{n^2} \sum\limits_{i=2}^{\infty} \frac{1}{ 2^i} \overset{1}{=} \frac{1}{2}\frac{1}{n^2}\Rightarrow \sum\limits_{i=2}^{\infty} \frac{1}{n^i i!} = O\left(\frac{1}{n^2}\right)$$
Gdzie $1$ wynika ze wzoru na sumę szeregu geometrycznego.
W takim razie mamy:
$$e^{1/n}= 1+ \frac{1}{n}+O\left(\frac{1}{n^2}\right)$$
\end{proof}

\subsection*{Zadanie 7}
Rozważmy algorytm sortujący n liczb w następujący sposób. Wybierz najmniejszą, postaw na pierwszym miejscu, wybierz najmniejszą z pozostałych i postaw na drugim miejscu, najmniejszą z pozostałych postaw na trzecim miejscu itd. aż do wyczerpania liczb. Określ złożoność czasową powyższej procedury.

Rozważmy najgorszy możliwy przypadek. Wtedy, w pierwszym kroku wykonujemy $n-1$ porównań. W drugim $n-2$, w trzecim $n-3$, aż w końcu w $n$tym wykonujemy $0$ porównań. Ile porównań wykonaliśmy w sumie?
$$(n-1)+(n-2)+(n-3)+\dots+0=\sum\limits_{i=0}^{n-1} i = \frac{1}{2}(n-1)n = O(n^2)$$
Złożoność to $O(n^2)$.

\subsection*{Zadanie 8}
Oceń złożoność czasową pisemnego dodawania i mnożenia liczb długości $n$.

\begin{itemize}
\item Dodawanie
Dodajemy do siebie dwie liczby długości $n$. Dodajemy do siebie ich cyfry i ewentualne przeniesienie. Operacją jednostkową jest dodawanie. Dla każdej z $n$ cyfr możemy więc wykonać trzy dodawania. W ten sposób otrzymujemy maksymalnie $2n$ dodawań, mamy więc złożoność $O(n)$.

\item Mnożenie
Mnożymy dwie liczby długości $n$. Dla każdej z $n$ liczb wykonujemy $n$ mnożeń i w najgorszym wypadku dodatkowo $n$ dodawań. Daje nam to $2n^2$ operacji. Następnie dodajemy do siebie $n$ liczb długości maksymalnie $2n-1$. To daje nam dodatkowo $2n^2-n$ operacji. W sumie mamy maksymalnie $4n^2-n$ operacji, czyli $O(n^2)$.

  
\end{itemize}

\subsection*{Zadanie 13}

Wykaż, że:

\begin{enumerate}[(a)]
\item W przedziale $[a,b]$ jest $\floor{b}-\ceil{a}+1$ liczb całkowitych.
\begin{proof}
Najmniejsza liczba całkowita w przedziale to $\ceil{a}$, a największa to $\floor{b}$.\\
Pomiędzy $\ceil{a}$ a $\floor{b}$ jest $\floor{b}-\ceil{a}-1$ liczb, dodatkowo $\ceil{a}$ oraz $\floor{b}$, to daje nam $\floor{b}-\ceil{a}+1$ liczb całkowitych.
\end{proof}

\item W przedziale $[a,b)$ jest $\ceil{b}-\ceil{a}$ liczb całkowitych.
\begin{proof}
Najmniejszą liczbą całkowitą w przedziale jest $\ceil{a}$.
Co do największej, to mamy dwa przypadki:
\begin{itemize}
\item Jeśli $b\in \mathbb{C}$ to największą liczbą w przedziale jest $b-1$, ale $b\in \mathbb{C} \Rightarrow b=\ceil{b}$. Skoro tak to w tym przedziale mamy $\ceil{b} - 1 - \ceil{a} + 1 = \ceil{b} - \ceil{a}$ liczb całkowitych w przedziale. 
\item Jeśli $b\notin \mathbb{C}$ to największą liczbą w przedziale jest $\floor{b}$, ale wiemy, że $b\notin \mathbb{C} \Rightarrow  \floor{b}=\ceil{b}-1$, mamy więc $\floor{b}-\ceil{a}+1=\ceil{b}-\ceil{a}$ liczb całkowitych w przedziale. 
\end{itemize}
\end{proof}


\item W przedziale $(a,b]$ jest $\floor{b}-\floor{a}$ liczb całkowitych.
\begin{proof}
Największą liczbą całkowitą w przedziale jest $\floor{b}$.
Co do najmniejszej, to mamy dwa przypadki:
\begin{itemize}
\item Jeśli $a\in \mathbb{C}$ to najmniejszą liczbą w przedziale jest $a+1$, ale $a\in \mathbb{C} \Rightarrow a=\floor{a}$. Skoro tak to w tym przedziale mamy $\floor{b} - (\floor{a} + 1) + 1 = \floor{b} - \floor{a}$ liczb całkowitych w przedziale. 
\item Jeśli $a\notin \mathbb{C}$ to najmniejszą liczbą w przedziale jest $\ceil{a}$, ale wiemy, że $a\notin \mathbb{C} \Rightarrow \ceil{a}=\floor{a}+1$, mamy więc $\floor{b}-\ceil{a}+1=\floor{b}-\floor{a}$ liczb całkowitych w przedziale. 
\end{itemize}
\end{proof}
\clearpage
\item W przedziale $(a,b)$ jest $\ceil{b}-\floor{a}-1$ liczb całkowitych.

\begin{proof}
Mamy cztery przypadki (choć bardzo podobne, zadbajmy jednak o jakiś formalizm):
\begin{itemize}
\item $a,b \in \mathbb{C}$\\
Najmniejszą liczbą jest $a+1=\floor{a}+1$. Największą jest $b-1=\ceil{b}-1$. Mamy więc $\ceil{b}-1 - \floor{a} - 1 + 1 = \ceil{b}-\floor{a} - 1$ liczb całkowitych w przedziale.

\item $a,b \notin \mathbb{C}$\\
Najmniejszą liczbą jest $\ceil{a}=\floor{a}+1$ (bo $a \notin \mathbb{C}$). Największą liczbą jest $\floor{b}=\ceil{b}-1$. Mamy więc $\ceil{b}-1-\floor{a}-1 + 1 = \ceil{b}-\floor{a}-1$ liczb całkowitych w przedziale.

\item $a \in \mathbb{C} \wedge b \notin \mathbb{C}$\\
Najmniejszą liczbą jest $a+1=\floor{a}+1$. Największą liczbą jest $\floor{b}=\ceil{b}-1$. Mamy więc $\ceil{b}-1 - \floor{a} - 1 + 1 = \ceil{b} - \floor{a} - 1$ liczb całkowitych w przedziale.

\item $a \notin \mathbb{C} \wedge b \in \mathbb{C}$\\
Najmniejszą liczbą jest $\ceil{a}=\floor{a}+1$.  Największą jest $b-1=\ceil{b}-1$. Mamy więc  $\ceil{b}-1 - \floor{a} - 1 + 1 = \ceil{b} - \floor{a} - 1$ liczb całkowitych w przedziale.

\end{itemize}
\end{proof}

\end{enumerate}



\end{document}
