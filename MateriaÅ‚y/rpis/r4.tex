\documentclass[a4paper]{article}
\usepackage[left=3cm,right=3cm,top=2cm,bottom=2cm]{geometry} % page settings
\usepackage{enumerate}
\usepackage{hyperref}
\usepackage{graphicx}
\usepackage{amsfonts}
\usepackage{amsthm}
\usepackage{mathtools}
\usepackage{titlesec}
\usepackage{polski}
\usepackage{tikz}
\usepackage[utf8]{inputenc}
\DeclarePairedDelimiter\ceil{\lceil}{\rceil}
\DeclarePairedDelimiter\floor{\lfloor}{\rfloor}
\DeclarePairedDelimiter\set{\lbrace}{\rbrace}


\def\checkmark{\tikz\fill[scale=0.3](0,.35) -- (.25,0) -- (1,.7) -- (.25,.15) -- cycle;} 

\titlespacing*{\subsection}
{0ex}{10ex}{3ex}

\title{Lista 4}
\author{Kamil Matuszewski}
\date{\today}

\begin{document}

\maketitle
\setlength{\parindent}{0.5ex}
\setlength{\parskip}{1.5ex}
\newcommand{\R}{\mathbb{R}}
\newcommand{\N}{\mathbb{N}}


\begin{center}
\begin{tabular}{|c *{10}{|c} |c|}\hline
1 & 2 & 3 & 4 & 5 & 6 & 7 & 8 & 9 & 10 & 11 & 12\\
\hline 
 & & & & & & & & & & & \\
\hline
\end{tabular}\\
\end{center}

\subsection*{Zadanie 1}
Zmienna losowa X ma gęstość o wzorze $f(x)=a+bx^2$ dla $0\leq x\leq 1$. Wiadomo też, że $E(X)=0,6$. Znajdź $a$ i $b$.

Wiemy, że $\int\limits_{-\infty}^\infty f(x) = 1 dx$ oraz, że $\int\limits_{-\infty}^\infty xf(x) dx = 0,6$. Pozostają więc tylko obliczenia:

$$\int\limits_0^1 a+bx^2 dx = a + \frac{b}{3} = 1 \Rightarrow a=1-\frac{b}{3}$$.
$$\int\limits_0^1 (1-\frac{b}{3})x+bx^3 dx = \int\limits_0^1 x-b(\frac{x}{3}+x^3) dx = \frac{1}{2}-b\int\limits_0^1 \frac{x}{3}+x^3 dx = \frac{1}{2}-b(\frac{1}{6}+\frac{1}{4} )$$
$$\frac{1}{2}-\frac{5}{12}b = \frac{3}{5} \Rightarrow b=-\frac{6}{25}; a=\frac{27}{25}$$
$$f(x)=\frac{27}{25}-\frac{6}{25}x^2 $$
Dodatnie na przedziale $[0,1]$, więc chyba działa.

\subsection*{Zadanie 2}
Oblicz wartość oczekiwaną rozkładu $Exp(\lambda)$ tzn takiego, gdzie $f(x)=\lambda e^{-\lambda x}$.

$$\int\limits_0^\infty x \lambda e^{-\lambda x} dx = \lambda \int\limits_0^\infty x e^{-\lambda x} dx = \lambda \int\limits_0^\infty x \left( \frac{-e^{-\lambda x}}{\lambda}\right)' dx = \lambda \left( \left[ x\frac{-e^{-\lambda x}}{\lambda} \right]_0^\infty - \int\limits_0^\infty \frac{-e^{-\lambda x}}{\lambda} dx \right)$$
$$= \lambda \left( 0 + \frac{1}{\lambda} \int\limits_0^\infty e^{-\lambda x} dx \right)= \int\limits_0^\infty e^{-\lambda x} dx = \left[ -\frac{e^{-\lambda x}}{\lambda}\right]_0^\infty = \frac{1}{\lambda}$$

\subsection*{Zadanie 3}
Oblicz wariancje rozkładu $Exp(\lambda)$ tzn takiego, gdzie $f(x)=\lambda e^{-\lambda x}$.

$Var(X)=E(X^2)-(E(X))^2$
Policzmy więc $E(X^2)$. W sumie liczyliśmy to w zadaniu poprzednim, z tą różnicą, że tym razem mamy $x^2$. Stąd niektóre przejścia są nierozpisane - są analogicze do tych z zad. 2.

$$\int\limits_0^\infty x^2 \lambda e^{-\lambda x} dx = \lambda \int\limits_0^\infty x^2 e^{-\lambda x} dx = \lambda \left( \left[ x^2\frac{-e^{-\lambda x}}{\lambda} \right]_0^\infty - \int\limits_0^\infty 2x \frac{-e^{-\lambda x}}{\lambda} dx \right)=$$
$$=2\int\limits_0^\infty x (e^{-\lambda x}) dx = \frac{2}{\lambda^2}$$

Ostatnie przejście wynika z zad 2 - liczymy to samo, tylko tym razem nie skróci się nam $\frac{1}{\lambda}$

$E(X)$ mamy już policzone, wystarczy podnieść do kwadratu. Stąd:

$$Var(X)=E(X^2)-(E(X))^2 = \frac{2}{\lambda^2} - \frac{1}{\lambda^2} = \frac{1}{\lambda^2}$$

\subsection*{Zadanie 4}
Oblicz wartość oczekiwaną zmiennej X, o gęstości $f(x)=xe^{-x} $

$$\int\limits_0^\infty x^2 e^{-x} = \int\limits_0^\infty x^2 (-e^{-x})' = \left[ x^2(-e^{-x}) \right]_0^\infty + 2\int\limits_0^\infty x e^{-x} = 2\int\limits_0^\infty x (-e^{-x})'$$ 
$$=2\left( \left[ x(-e^{-x}) \right]_0^\infty + \int\limits_0^\infty e^{-x} \right) = 2$$

\subsection*{Zadanie 5}
Oblicz wartość oczekiwaną zmiennej X, o gęstości
$$f(x)=\left\{\begin{matrix}
x & dla  & 0\leq x\leq 1 \\ 
2-x & dla  & 1\leq x\leq 2 \\ 
0 & w.p.p  & 
\end{matrix}\right.$$

$$\int\limits_0^1 x^2 dx + \int\limits_1^2 2x-x^2 dx = \int\limits_0^1 x^2 dx + 2\int\limits_1^2 x dx - \int\limits_1^2 x^2 dx = \frac{1}{3} + 3 - \frac{7}{3} = 1$$

\subsection*{Zadanie 6}
Oblicz wartość oczekiwaną zmiennej X, o dystrybuancie
$$F(x)=\left\{\begin{matrix}
0 & dla  & 0\leq x\leq 1 \\ 
2-\frac{2}{x} & dla  & 1\leq x\leq a \\ 
1 & dla  & x>a 
\end{matrix}\right.$$

Najpierw, gęstość. W tym celu liczymy pochodne.
$$f(x)=\left\{\begin{matrix}
0 & dla  & 0\leq x\leq 1 \\ 
\frac{2}{x^2} & dla  & 1\leq x\leq a \\ 
0 & dla  & x>a 
\end{matrix}\right.$$

Teraz, liczymy a:

$$\int\limits_1^a \frac{2}{x^2} dx = 2\int\limits_1^a \frac{1}{x^2} dx = \left[-\frac{2}{x}\right]_0^a $$
$$\left[-\frac{2}{x}\right]_0^a = 1 \Rightarrow a=2$$

Więc nasza gęstość to $\frac{2}{x^2}$ dla $x\in [1,2]$. Wartość oczekiwana to już tylko formalność:

$$\int\limits_1^2 \frac{2}{x} dx = \left[2log(x)\right]_1^2 = log(4)$$

\subsection*{Zadanie 7}
Oblicz wartość oczekiwaną zmiennej X o gęstości $f(x)=3x^2$ dla $x\in [0,1]$

$$\int_0^1 3x^3 dx = \frac{3}{4} $$


\end{document}
