\documentclass[a4paper]{article}
\usepackage[left=3cm,right=3cm,top=2cm,bottom=2cm]{geometry} % page settings
\usepackage{enumerate}
\usepackage{hyperref}
\usepackage{graphicx}
\usepackage{amsfonts}
\usepackage{amsthm}
\usepackage{mathtools}
\usepackage{titlesec}
\usepackage{polski}
\usepackage{tikz}
\usepackage[utf8]{inputenc}
\DeclarePairedDelimiter\ceil{\lceil}{\rceil}
\DeclarePairedDelimiter\floor{\lfloor}{\rfloor}
\DeclarePairedDelimiter\set{\lbrace}{\rbrace}


\def\checkmark{\tikz\fill[scale=0.3](0,.35) -- (.25,0) -- (1,.7) -- (.25,.15) -- cycle;} 

\titlespacing*{\subsection}
{0ex}{10ex}{3ex}

\title{Lista 5}
\author{Kamil Matuszewski}
\date{\today}

\begin{document}

\maketitle
\setlength{\parindent}{0.5ex}
\setlength{\parskip}{1.5ex}
\newcommand{\R}{\mathbb{R}}
\newcommand{\N}{\mathbb{N}}


\begin{center}
\begin{tabular}{|c *{9}{|c} |c|}\hline
1 & 2 & 3 & 4 & 5 & 6 & 7 & 8 & 9 & 10 & 11\\
\hline 
\checkmark &\checkmark &\checkmark &\checkmark &\checkmark &\checkmark &&&&\checkmark &\checkmark \\
\hline
\end{tabular}\\
\end{center}

\subsection*{Zadanie 1}
Znajdź MGF dla zmiennej $X \sim B(n,p)$.

Po pierwsze gęstość $X \sim B(n,p)$ to ${n \choose k}p^k q^{n-k}$. Po drugie, MFG $M(t)=E(e^{tX})$. Podstawiamy i mamy:
$$E(e^{tX}) = \sum\limits_{k=0}^n e^{tk}{n \choose k}p^k q^{n-k} = \sum\limits_{k=0}^n {n \choose k}(pe^t)^k q^{n-k} = (pe^t + q)^n = (pe^t + 1-p)^n$$


\subsection*{Zadanie 2}
Znajdź MFG dla zmiennej $X \sim Poisson(\lambda)$.

To samo co wyżej. Gęstość $X \sim Poisson(\lambda)$ to $f(x) = \frac{\lambda^x e^{-\lambda}}{x!}$. Podstawiamy i mamy:
$$M(t)=\sum_{k=0}^{\infty} e^{tk}\frac{\lambda^k e^{-\lambda}}{k!} = e^{-\lambda}\sum_{k=0}^{\infty} \frac{(e^t\lambda)^k}{k!} = e^{-\lambda+\lambda e^t} =e^{\lambda(e^t-1)}$$

\subsection*{Zadanie 3}
Znajdź MGF dla zmiennej $X \sim Gamma(b,p)$

To samo. Gęstość $X \sim Gamma(b,p)$ to $f(x)=\frac{b^p}{\Gamma(p)}x^{p-1}e^{-bx}$. Podstawiamy, i mamy:
$$M(t)=\int_0^\infty e^{tx} \frac{b^p}{\Gamma(p)}x^{p-1}e^{-b x} dx = \int_0^\infty \frac{b^p}{\Gamma(p)}x^{p-1}e^{(t-b) x} dx =\left\{\begin{matrix}
y=(b-t)x & x=\frac{y}{b-t} \\ 
dx=\frac{1}{b - t}dy & 
\end{matrix}\right.$$
$$ M(t)=\int_0^\infty \frac{b^p}{\Gamma(p)}x^{p-1}e^{-y}\frac{1}{\lambda - t} dy = \frac{b^p}{\Gamma(p)(b-t)^p}\int_0^\infty y^{p-1}e^{-y} dy = \frac{b^p}{\Gamma(p)(b-t)^p} \Gamma(p) = \left(\frac{b}{b-t}\right)^p$$

\subsection*{Zadanie 4}
Mamy zmienną $Z=\sum\limits_{k=1}^\infty X_k$. Podać rozkład Z dla $X_k \sim N(\mu_k, \sigma_k^2)$\\

Po pierwsze, wyznaczmy MFG rozkładu normalnego. Gęstość rozkładu normalnego to $\frac{1}{\sqrt{2\pi \sigma^2 }} e^{-\frac{1}{2}(x-\mu)^2/\sigma^2}$
$$M(t)=\int e^{xt} \frac{1}{\sqrt{2\pi \sigma^2 }} e^{-\frac{1}{2}(x-\mu)^2/\sigma^2} dx = \left\{\begin{matrix}
z=\frac{x-\mu}{\sigma} & x=z\sigma + \mu \\ 
dx=\sigma dz & 
\end{matrix}\right.$$

$$M(t)=e^{\mu t}\int e^{z\sigma t} \frac{1}{\sigma \sqrt{2\pi}}e^{-\frac{1}{2}z^2} \sigma dz = e^{\mu t}\int e^{z\sigma t} \frac{1}{\sqrt{2\pi}}e^{-\frac{1}{2}z^2} dz = *$$
Teraz:
$$\int e^{z t} \frac{1}{\sqrt{2\pi}}e^{-\frac{1}{2}z^2} dz $$
$$e^{zt}e^{-\frac{1}{2}z^2} = e^{-\frac{1}{2}z^2 +zt} = e^{-\frac{1}{2}(z-t)^2}e^{\frac{1}{2}t^2}$$
$$e^{\frac{1}{2}t^2} \int \frac{1}{\sqrt{2\pi}}e^{-\frac{1}{2}(z-t)^2} dz = e^{\frac{1}{2}t^2} \int_\R N(t,1) = e^{\frac{1}{2}t^2} $$
Ostatnia równość stąd, że to całka po gęstości która zawsze jest 1 (z definicji).
$$*=e^{\mu t}e^{\frac{1}{2}\sigma^2 t^2} $$

Wróćmy do zadania. Wiemy, że dla  $Z=X_1 + X_2 + \dots + X_n $ jeśli zmienne są niezależne, to zachodzi $M_Z(t)=M_{X_1}(t)M_{X_2}(t)\dots M_{X_n}(t)$. Stąd, mamy:
$$M_Z(t)=M_{X_1}(t)M_{X_2}(t)\dots M_{X_n}(t) = e^{\mu_1 t}e^{\frac{1}{2}\sigma_1^2 t^2} e^{\mu_2 t}e^{\frac{1}{2}\sigma_2^2 t^2}\dots e^{\mu_n t}e^{\frac{1}{2}\sigma_n^2 t^2} = exp \set{t \sum\limits_{k=1}^n \mu_k} \set{\frac{1}{2} t^2 \sum\limits_{k=1}^n \sigma_n^2 }$$
Oznaczając $a=\sum\limits_{k=1}^n \mu_k$ i $b= \sum\limits_{k=1}^n \sigma_n^2$ otrzymujemy funkcję generującą momenty dla rozkładu $N(a,b)$

\subsection*{Zadanie 5}
Mamy zmienną $Z=\sum\limits_{k=1}^\infty X_k$. Podać rozkład Z dla $X_k \sim Gamma(b, p_k)$\\

Tu już mamy funkcję generującą momenty. Liczyliśmy ją z w zadaniu 3. Podobnie jak w poprzednim:
$$M_Z(t)=M_{X_1}(t)M_{X_2}(t)\dots M_{X_n}(t) = \left(\frac{b}{b-t}\right)^p_1 \left(\frac{b}{b-t}\right)^p_2 \dots \left(\frac{b}{b-t}\right)^p_k$$
Więc, dla $a=\sum_{k=1}^n p_k$ mamy:
$$M_Z(t) = \left(\frac{b}{b-t}\right)^a $$
Stąd otrzymujemy funkcję generującą momenty dla rozkładu $Gamma(b,a)$

\subsection*{Zadanie 6}
Mamy zmienną $Z=\sum\limits_{k=1}^\infty X_k$. Podać rozkład Z dla $X_k \sim B(m_k, p)$\\

Mamy już funkcję generującą momenty z zad 1. Podobnie jak w poprzednich:\\
$$M_Z(t)=M_{X_1}(t)M_{X_2}(t)\dots M_{X_n}(t) = (pe^t + 1-p)^{m_1} (pe^t + 1-p)^{m_2} \dots (pe^t + 1-p)^{m_n}$$
Więc, dla $a=\sum_{k=1}^n m_k$ mamy:
$$M_Z(t)=(pe^t + 1-p)^a$$
Stąd otrzymujemy funkcję generującą momenty dla rozkładu $B(a,p)$.

\subsection*{Zadanie 10}
Mamy MGF $M_U(t) = \frac{2}{2-3t}$. Wyznacz:
\begin{enumerate}[a)]
\item Wartość oczekiwaną:\\
To po prostu pochodna - $M'_U(t)=\left(\frac{2}{2-3t}\right)' = \frac{6}{(2-3t)^2}$, $M'_U(0)=\frac{6}{4}$.
\item Wariancje:\\
Druga pochodna w punkcie 0 - $E(X^2)=\frac{36}{8}$ odjąć $(E(X))^2 = \frac{36}{16}$, stąd $V(X)=\frac{36}{8} - \frac{36}{16} = \frac{9}{4}$
\item MGF podatku od zysku, przy założeniu stopy podatkowej liniowej, $90\%$\\
Niech $Y=0,9U$. Wiemy, że $M_{aX}(t)=M_X(at)$. Stąd, wynik to $\frac{2}{2-2,7t}$
\end{enumerate}
\subsection*{Zadanie 11}
Mamy MFG zmiennej losowej X. Co można powiedzieć o Y, gdy:

\begin{enumerate}[a)]
\item $M_Y(t)=M(2t)M(4t)$\\
$M_{2X_1 + 4X_2}(t) = M_{2X_1}(t)M_{4X_2}(t) = M(2t)M(4t)$ jeśli $X_1$ i $X_2$ są niezależne.
\item $M_Y(t)=e^{2t}M(t)$\\
$M_{X+2}=E(e^{(X+2)t}) = E(e^{Xt}e^{2t}) = \int e^{xt}e^{2t} f(x) = e^{2t}\int e^{xt}f(x)=e^{2t}E(e^{Xt})=e^{2t}M(t)$
\item $M_Y(t)=4M(t)$\\
Tu można coś pomachać, że dla dowolnej funkcji $M(0)=1$, więc $4M(0)=4$
\end{enumerate}



\end{document}
