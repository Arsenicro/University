\documentclass[a4paper]{article}
\usepackage[left=3cm,right=3cm,top=2cm,bottom=2cm]{geometry} % page settings
\usepackage{enumerate}
\usepackage{hyperref}
\usepackage{graphicx}
\usepackage{amsfonts}
\usepackage{amsthm}
\usepackage{mathtools}
\usepackage{titlesec}
\usepackage{polski}
\usepackage{tikz}
\usepackage[utf8]{inputenc}
\DeclarePairedDelimiter\ceil{\lceil}{\rceil}
\DeclarePairedDelimiter\floor{\lfloor}{\rfloor}
\DeclarePairedDelimiter\set{\lbrace}{\rbrace}


\def\checkmark{\tikz\fill[scale=0.3](0,.35) -- (.25,0) -- (1,.7) -- (.25,.15) -- cycle;} 

\titlespacing*{\subsection}
{0ex}{10ex}{3ex}

\title{Lista 1}
\author{Kamil Matuszewski}
\date{\today}

\begin{document}

\maketitle
\setlength{\parindent}{0.5ex}
\setlength{\parskip}{1.5ex}
\newcommand{\R}{\mathbb{R}}
\newcommand{\N}{\mathbb{N}}


\begin{center}
\begin{tabular}{|c *{11}{|c} |c|}\hline
1 & 2 & 3 & 4 & 5 & 6 & 7 & 8 & 9 & 10 & 11 & 12 & 13-14\\
\hline 
\checkmark &\checkmark &\checkmark &\checkmark &\checkmark &\checkmark &\checkmark &\checkmark &\checkmark &\checkmark &\checkmark &\checkmark & \\
\hline
\end{tabular}\\
\end{center}

\subsection*{Zadanie 1}

Niech $\Sigma$ będzie $\sigma$-ciałem zbiorów.\\
\begin{enumerate}[(a)]
\item Sprawdzić, że $\Omega \in \Sigma$\\
Wiemy, że $\emptyset \in \Sigma$, oraz, że $A \in \Sigma \Rightarrow A^C \in \Sigma$ (Def 2). Stąd:\\
$$\emptyset \in \Sigma \Rightarrow \emptyset^C=\Omega \in \Sigma$$
\item Załóżmy, że $A_k \in \Sigma$, dla $k=1,2,\dots$. Wykazać, że $\bigcap\limits_{k \in \N} A_k$ $\in \Sigma$
\begin{proof}
Niech $A^C = \Omega \slash A$\\
Z praw de Morgana:
$$\bigcap\limits_k A_k = \Omega \slash \left( \Omega \slash \bigcap\limits_k A_k \right) =  \Omega \slash \left(\bigcup\limits_k  \Omega \slash A_k \right) $$
Teraz, wiemy, że $A_k \in \Sigma$ (z założenia). Z def 2.2 wiemy, że $\Omega \slash A_k \in \Sigma$. Z def 2.3 wiemy, że w takim razie $\bigcup\limits_k  \Omega \slash A_k \in \Sigma$. Ponownie z def 2.2 $\Omega \slash \left(\bigcup\limits_k  \Omega \slash A_k \right) = \bigcap\limits_k A_k \in \Sigma$.
\end{proof}
\end{enumerate}


\subsection*{Zadanie 2}
Znajdź wszystkie $\sigma$-ciała dla $\Omega = \set{a,b,c}$.
$$\Sigma_1 = \Big\{ \emptyset, \Omega, \set{a}, \set{b,c} \Big\}$$
$$\Sigma_2 = \Big\{ \emptyset, \Omega, \set{b}, \set{a,c} \Big\}$$
$$\Sigma_3 = \Big\{ \emptyset, \Omega, \set{c}, \set{a,b} \Big\}$$
$$\Sigma_4 = \Big\{ \emptyset, \Omega \Big\}$$
\clearpage
\subsection*{Zadanie 3}
Znajdź najmniejsze $\sigma$-ciało zawierające zbiór $S=\set{1,3}$ dla $\Omega = \set{1,2,3,4,5}$.

$$\Sigma = \Big\{ \emptyset, \Omega, \set{1,3}, \set{2,4,5} \Big\}$$

\subsection*{Zadanie 4}
Niech $\Omega = \set{a,b,c}$. Podać przykład funkcji X,Y takich, że X jest zmienną losową a Y nie jest zmienną losową.

Niech X: $a\rightarrow 1, b\rightarrow 2, c\rightarrow 2$\\
Y: $a\rightarrow 2, b\rightarrow 1, c\rightarrow 2$\\
$\Sigma = \Big\{ \emptyset, \Omega, \set{a}, \set{b,c} \Big\}$

Wtedy, dla przestrzeni probabilistycznej $(\Omega, \Sigma, P)$ mamy:
$$X^{-1}((-\infty,a])=\set{a} \in \Sigma$$
$$Y^{-1}((-\infty,a])=\set{b} \notin \Sigma$$

\subsection*{Zadanie 5}
Znajdź funkcję prawdopodobieństwa i wartość oczekiwaną EX zmiennej losowej X, której dystrybuanta określona jest następująco:
\begin{tabular}{*{5}{c}}
x & $(-\infty; -2]$ & $(-2,3]$ & $(3,5]$ & $(5,\infty)$\\
F(x) & 0 & 0,2 & 0,7 & 1
\end{tabular}

Wypiszmy punkty: 5,3,-2.
Teraz $p_i$: 1-0,7=0,3; 0,7-0,2=0,5; 0,2-0=0,2. Stąd:

\begin{tabular}{*{4}{c|}}
$x_i$ & $-2$ & $3$ & $5$\\ \hline
$p_i$ & 0,2 & 0,5 & 0,3
\end{tabular}

Co do $EX=\sum\limits_{i=0}^2 x_i\cdot p_i = -2\cdot 0,2 + 3\cdot 0,5 + 5\cdot 0,3 = 2,6$

\subsection*{Zadanie 6}
Zmienna X ma rozkład Bernoulliego z parametrami $n, p (X \sim B(n, p))$. Sprawdzić, że: 
\begin{itemize}
\item $\sum\limits_{k=0}^n p_k = \sum\limits_{k=0}^n {n \choose k} p^k (1-p)^{n-k} = 1 $
\begin{proof}
Szansa na uzyskanie k sukcesów po n próbach to ${n \choose k} p^k (1-p)^{n-k}$ bo mamy k zwycięstw z prawdopodobieństwem sukcesu p, i n-k porażek z prawdopodobieństwem 1-p, a do tego wybieramy za którym razem ma być sukces. Stąd nasz wzór to $\sum\limits_{k=0}^n p_k = \sum\limits_{k=0}^n {n \choose k} p^k (1-p)^{n-k}$
$$\sum\limits_{k=0}^n {n \choose k} p^k (1-p)^{n-k} \stackrel{dwumian newtona}{=} (1-p+p)^n = 1^n =1  $$
\end{proof}
\item $E(X)=\sum\limits_{k=0}^n k {n \choose k} p^k (1-p)^{n-k}=np$
\begin{proof}
Dla $x_k=k$, i $E(X)=\sum\limits_{k=0}^n x_k p_k = \sum\limits_{k=0}^n k p_k = \sum\limits_{k=0}^n k {n \choose k} p^k (1-p)^{n-k}$ Stąd mamy:
$$\sum\limits_{k=0}^n k {n \choose k} p^k (1-p)^{n-k} = \sum\limits_{k=1}^n k \frac{n!}{k!(n-k)!} p^k (1-p)^{n-k} = \sum\limits_{k=1}^n np \frac{(n-1)!}{(k-1)!(n-k)!} p^{k-1} (1-p)^{n-k} =$$ $$=np \sum\limits_{k=1}^n {n-1 \choose k-1} p^{k-1} (1-p)^{n-k} = np \sum\limits_{k=0}^{n-1} {n-1 \choose k} p^{k} (1-p)^{n-k-1} \stackrel{poprzedni}{=} np \cdot 1 = np$$ 
\end{proof}
\end{itemize}

\subsection*{Zadanie 7}
Zmienna X ma rozkład Poissona z parametrem $\lambda$. Sprawdzić, że:
\begin{itemize}
\item $\sum\limits_{k=0}^\infty p_k = \sum\limits_{k=0}^\infty e^{-\lambda} \frac{\lambda^k}{k!} = 1$
\begin{proof}
$\sum\limits_{k=0}^\infty e^{-\lambda} \frac{\lambda^k}{k!} =  e^{-\lambda} \sum\limits_{k=0}^\infty \frac{\lambda^k}{k!} = e^{-\lambda} e^\lambda=1$
\end{proof}
\item $E(X)=\sum\limits_{k=0}^\infty k e^{-\lambda} \frac{\lambda^k}{k!} = \lambda$ 
\begin{proof}

$$\sum\limits_{k=0}^\infty k e^{-\lambda} \frac{\lambda^k}{k!} = \sum\limits_{k=1}^\infty e^{-\lambda} \lambda \frac{\lambda^{k-1}}{(k-1)!} = e^{-\lambda} \lambda \sum\limits_{k=1}^\infty \lambda \frac{\lambda^k}{(k-1)!} = e^{-\lambda} \lambda \sum\limits_{k=0}^\infty \frac{\lambda^{k}}{k!}=e^{-\lambda} e^\lambda \lambda = \lambda$$
\end{proof}
\end{itemize}

\subsection*{Zadanie 8}
Udowodnić, że $E(aX+b)=aE(X)+b$
\begin{proof}
$$E(aX+b)=\sum\limits_{k=0} (a x_k + b) p_k = \sum\limits_{k=0} a x_k p_k + \sum\limits_{k=0} b p_k = a \sum\limits_{k=0} x_k p_k + b \sum\limits_{k=0}p_k$$
Wiemy, że  $E(X)=\sum\limits_{k=0} x_k p_k$ oraz, że $\sum\limits_{k=0}p_k = 1$, stąd:
$$E(aX+b) = a E(X) + b\cdot 1 =  a E(X) + b $$
\end{proof}

\subsection*{Zadanie 9}
Udowodnić, że $V(X)=E(X^2)-(EX)^2$.
\begin{proof}
$$V(X)=E(X-EX)^2=\sum\limits_k (x_k-EX)^2 p_k = \sum\limits_k (x_k^2-2x_kEX + (EX)^2) p_k =$$ $$= \sum\limits_k x_k^2 p_k - 2EX\sum\limits_k x_k p_k + (EX)^2 \sum\limits_k p_k =EX^2-2EX\cdot EX + (EX)^2 \cdot 1 =$$ $$= EX^2-2(EX)^+(EX)^2=EX^2-(EX)^2$$
\end{proof}

\subsection*{Zadanie 10}
Pokaż, że $V(aX + b) = a^2 V(X)$.\\
\begin{proof}
$$V(aX+b)\stackrel{zad 9}{=}E[(aX+b)^2]-(E[aX+b])^2=E[a^2X^2+2abX+b^2]-(aE(X)+b)^2=$$ $$=a^2E(X^2)+2abE(X)+b^2-a^2E^2(X)-2abE(X)-b^2=$$ $$=a^2E(X^2)-a^2E^2(X)=a^2(EX^2-(EX)^2)=a^2V(X) $$
\end{proof}

\subsection*{Zadanie 11}
Wykaż, że $\Gamma(n)=(n-1)!$, $n \in \N$
\begin{proof}
$$\Gamma(n)=\int\limits_0^\infty t^{n-1}e^{-t} dt$$
$$\Gamma(1)=\int\limits_0^\infty 1e^{-t} dt = 1$$
$$\Gamma(n)=\int\limits_0^\infty t^{n-1}e^{-t} dt = \int\limits_0^\infty t^{n-1}(-e^{-t})' dt = \left[t^{n-1}e^{-t}\right]_0^\infty - \int\limits_0^\infty -(n-1)t^{n-2}e^{-t} dt = \left[t^{n-1}e^{-t}\right]_0^\infty + \int\limits_0^\infty (n-1)t^{n-2}e^{-t} dt$$
Skoro $e^{-\infty}=0$ i $0^{n-1}=0$, stąd pierwsze wyrażenie to $0$.
$$\Gamma(n)=\int\limits_0^\infty (n-1)t^{n-2}e^{-t} dt = (n-1)\int\limits_0^\infty t^{n-2}e^{-t} dt=(n-1)\Gamma(n-1)$$
Stąd, oraz z tego, że $\Gamma(1)=1$, mamy:
$$\Gamma(n)=1\cdot 2\cdot 3 \cdots (n-1) = (n-1)! $$
\end{proof}

\subsection*{Zadanie 12}
Sprawdź, że:
\begin{enumerate}[(a)]
\item $B(p, q + 1) = \frac{q}{p+q} B(p,q)$
\begin{proof}
Najpierw, $t^p=t^{p-1}-t^{p-1}(1-t)$. Teraz:
$$B(p,q+1)=\int_0^1 t^{p-1}(1-t)^{q} dt= \left[\frac{t^p}{p}(1-t)^q\right]_0^1 + \frac{q}{p}\int_0^1 t^p(1-t)^{q-1} dt = \frac{q}{p} \int_0^1 t^{p-1}(1-t)^{q-1} - t^{p-1}(1-t)^q dt =$$ $$=\frac{q}{p}\left[ \int_0^1 t^{p-1}(1-t)^{q-1} dt - \int_0^1 t^{p-1}(1-t)^{q} dt  \right] = \frac{q}{p} \left[ B(p,q) - B(p,q+1)  \right]$$
Teraz:
$$B(p,q+1) = \frac{q}{p} B(p,q) - \frac{q}{p} B(p,q+1) $$
$$B(p,q+1) \frac{q+p}{p} = \frac{q}{p} B(p,q) $$
$$B(p,q+1) = \frac{q}{p}\cdot \frac{p}{q+p} B(p,q) $$
$$B(p,q+1) = \frac{q}{q+p} B(p,q) $$
\end{proof}

\item $B(p,q)=B(p,q+1) + B(p+1,q)$

\textbf{Lemat 1:} $B(p,q)=B(q,p)$.
\begin{proof}
$$B(p,q)=\int_0^1 t^{p-1} (1-t)^{q-1} dt = \left\{\begin{matrix}
1-t=x &\Rightarrow & t=1-x\\
dt=-dx
\end{matrix}\right. $$
Zastosujemy podstawienie. W tym podstawieniu, jeśli $t=0$ to $x=1$, a jeśli $t=1$ to $x=0$, stąd odwrócą się nam granice całkowania. Ale $dt=-dx$, a minus ponownie odwróci nam granice całkowania, stąd, granice całkowania pozostają bez zmian. Mamy więc:
$$B(p,q)=\int_0^1 x^{q-1}(1-x)^{p-1} dx = B(q,p) $$
\end{proof}


\textbf{Lemat 2:} $B(p+1,q)=\frac{p}{p+q} B(p,q)$.
\begin{proof}
Wynika to wprost z podpunktu (a) i Lematu 1: $$B(p+1,q)=B(q,p+1)=\frac{p}{p+q} B(q,p)=\frac{p}{p+q} B(p,q)$$
\end{proof}
Teraz, dowód właściwy:
\begin{proof}
$$B(p,q+1) + B(p+1,q) = \frac{q}{p+q} B(p,q) + \frac{p}{p+q} B(p,q) = \frac{q+p}{p+q} B(p,q) = B(p,q)$$
\end{proof}

\end{enumerate}

\end{document}
