\documentclass[a4paper]{article}
\usepackage[left=3cm,right=3cm,top=2cm,bottom=2cm]{geometry} % page settings
\usepackage{enumerate}
\usepackage{hyperref}
\usepackage{graphicx}
\usepackage{amsfonts}
\usepackage{mathtools}
\usepackage{titlesec}
\usepackage{polski}
\usepackage[utf8]{inputenc}
\DeclarePairedDelimiter\ceil{\lceil}{\rceil}
\DeclarePairedDelimiter\floor{\lfloor}{\rfloor}

\titlespacing*{\subsection}
{0ex}{10ex}{3ex}

\title{Lista 6}
\author{Kamil Matuszewski}
\date{\today}


\begin{document}


\maketitle
\setlength{\parindent}{0.5ex}
\setlength{\parskip}{1.5ex}

\subsection*{Zadanie 1}
Mamy algorytm w postaci:\\
$w_n=a_n$\\
$w_{n-1}=w_n x + a_{n-1}$\\
Mamy więc:\\
$a_0(1+\beta_0)+a_1 x(1+\alpha_1)(1+\beta_0)(1+\beta_1)+...+a_n x^n (1+\alpha_1)...(1+\alpha_n)(1+\beta_0)...(1+\beta_n)$\\
$=\sum\limits_{i=0}^n x^i a_i \prod\limits_{j=0}^i(1+\beta_j) \prod\limits_{j=1}^i(1+\alpha_j) $\\
Teraz, niech $(1+\beta)$ to będzie maksymalny błąd $(1+\beta_i)$, a $(1+\alpha)$ maksymalny błąd $(1+\alpha_i)$. Mamy wtedy:\\
$\sum\limits_{i=0}^n x^i a_i \prod\limits_{j=0}^i(1+\beta) \prod\limits_{j=1}^i(1+\alpha) $\\
$\sum\limits_{i=0}^n x^i a_i (1+\beta)^i (1+\alpha)^i $\\
A teraz $(1+\epsilon)=(1+\alpha)(1+\beta)$. Ten błąd jest nadal małym błędem, wtedy mamy:\\
\Large$\sum\limits_{i=0}^n x^i a_i ((1+\beta)(1+\alpha))^i = \sum\limits_{i=0}^n x^i a_i (1+\epsilon)^i = \sum\limits_{i=0}^n (x(1+\epsilon))^i a_i =\sum\limits_{i=0}^n \tilde{x}^i a_i $\\ \normalsize
a to jest dokładny wynik dla lekko zaburzonych danych, czyli algorytm jest numerycznie poprawny.

\clearpage
\subsection*{Zadanie 3}
\Large a) \normalsize

$T_{0}(x)=1$\\
$T_{1}(x)=x$\\
$T_{2}(x)=2x\cdot T_{1}(x) - T_{0}(x)$\\
$T_{2}(x)=2x^2 - 1$\\
$T_{3}(x)=2x\cdot T_{2}  - T_{1}$\\
$T_{3}(x)=2x\cdot (2x^2 - 1) - x = 4x^3-3x$\\
$T_{4}(x)=2x\cdot T_{3}  - T_{2}$\\
$T_{4}(x)=2x\cdot (4x^3-3x)  - 2x^2 + 1= 8x^4-6x^2-2x^2+1=8x^4-8x^2+1$\\
$T_{5}(x)=2x\cdot T_{4}  - T_{3}$\\
$T_{5}(x)=2x\cdot (8x^4-8x^2+1)  - 4x^3+3x = 16x^5-16x^3-4x^3+3x=16x^5-20x^3+5x$\\
$T_{6}(x)=2x\cdot T_{5}  - T_{4}$\\
$T_{6}(x)=2x\cdot (16x^5-20x^3+5x)  - 8x^4+8x^2-1 = 32x^6-48x^4+18x^2-1$

\Large b) \normalsize

Dla $n=1$ - oczywiste, z definicji.\\
Teraz załóżmy, że $\forall_{i<n}$ działa, sprawdźmy dla n.\\
Wiemy, że $T_{n}=2x\cdot T_{n-1} - T_{n-2}$, wiemy, że stopień n-tego wielomianu czebyszewa to n (łatwo można pokazać). Dlatego n-ta potęga powstanie przez przemnożenie 2x przez $T_{n-1}$, ale odjęcie $T_{n-2}$ nie wpłynie na współczynnik (bo jest stopnia n-2). Ale z założenia wiemy, że współczynnik przy $x^{n-1}$ w $T_{n-1}$ to $2^{n-2}$, mnożąc to przez 2, więc współczynnik to $2^{n-1}$, a to to co chcieliśmy pokazać.

\Large c) \normalsize

\large i)\\ \normalsize 
Wiemy, że $T_n(x)=\cos(n\cdot \arccos(x))$, a $\cos(\alpha) \in [-1,1]$, więc to jest co co mieliśmy pokazać...

\large ii)\\ \normalsize 
Z wiedzą, że $\cos(\alpha)=1 \leftrightarrow \alpha=k\cdot \pi$, oraz, że $T_n(x)=\cos(n\cdot \arccos(x))$, wiemy, że\\
$n\cdot \arccos(x) = k\cdot \pi$\\
$\arccos(x) = \frac{k\cdot \pi}{n}$\\
$x_k=\cos(\frac{k\cdot \pi}{n}), (k=0..n-1)$

\large iii)\\ \normalsize 
Chcemy, żeby $\cos((n+1)\arccos x) = 0$. $\cos(\alpha)=0$ gdy $\alpha=k\pi + \frac{\pi}{2}$. Więc:\\
\Large $ (n+1)\arccos x = k\pi + \frac{\pi}{2}$, więc $\arccos x = \frac{k\pi + 0.5\cdot \pi}{n+1} $  \\
$x_k=\cos(\frac{k\pi + 0.5\cdot \pi}{n+1}) $\\
\normalsize
Teraz wiemy, że istnieje co najwyżej n+1 zer (bo wielomian jest rzędu n+1). Weźmy $k=0...n$, jest ich n+1, więc jest ich przynajmniej n+1, skoro jest ich przynajmniej n+1 i najwyżej n+1 to jest ich dokładnie n+1.

\clearpage
\subsection*{Zadanie 4}
Załóżmy, że istnieją dwa wielomiany stopnia n interpolujące w węzłach $x_0...x_n$ te same wartości. Nazwijmy je $W_1(x)$ i $W_2(x)$
Weźmy teraz $W_3(x)=W_1(x)-W_2(x)$. Wiemy, z własności odejmowania wielomianów, że jest on stopnia co najwyżej n. Ale wiemy, że $W_1(x)$ i $W_2(x)$ w węzłach $x_0...x_n$ przyjmują te same wartości. Dla tych punktów mamy więc $W_3(x_k)=0$, ma on więc n+1 miejsc zerowych.\\
Ale $W_3(x)$ jest stopnia nie większego niż n, a każdy niezerowy wielomian stopnia n ma co najwyżej n pierwiastków rzeczywistych, co oznacza, że $W_3(x)$ musi być wielomianem tożsamościowo równy zeru. A to oznacza, że$ W_3(x)=W_1(x)-W_2(x)=0 \leftrightarrow W_1(x)=W_2(x)$, a to jest sprzeczne z założeniem, że są różne.

\subsection*{Zadanie 5}
$x_0=0, x_1=2, x_2=4, x_3=7$\\
$y_0=-1, y_1=1, y_2=3, y_3=-5$\\
\Large
$L_3(x)=(-1)\frac{(x-2)(x-4)(x-7)}{(0-2)(0-4)(0-7)}+(1)\frac{(x-0)(x-4)(x-7)}{(2-0)(2-4)(2-7)}+(3)\frac{(x-0)(x-2)(x-7)}{(4-0)(4-2)(4-7)}+(-5)\frac{(x-0)(x-2)(x-4)}{(7-0)(7-2)(7-4)}=$\\
$=\frac{(x-2)(x-4)(x-7)}{56}+\frac{(x-0)(x-4)(x-7)}{10}+\frac{(x-0)(x-2)(x-7)}{8}-\frac{(x-0)(x-2)(x-4)}{21}=\frac{41x^3-456x^2+1063x-210}{120}$
\normalsize

\subsection*{Zadanie 6}
a)\\
Wielomian stopnia $\leq$ 6 zinterpolować funkcję stopnia 3, więc z jednoznaczności z zad 4, to ta sama funkcja, koniec.\\
b)\\
$x_0=-1, x_1=0, x_2=1$\\
$y_0=1, y_1=1, y_2=5$\\
\Large
$L_2(x)=\frac{(x-0)(x-1)}{(-1-0)(-1-1)}+\frac{(x+1)(x-1)}{(0+1)(0-1)}+(5)\frac{(x+1)(x-0)}{(1+1)(1-0)}=\frac{(x-0)(x-1)}{2}-\frac{2(x+1)(x-1)}{2}+(5)\frac{(x+1)(x-0)}{2}=2x^2+2x+1$
\normalsize
 
\clearpage 
\subsection*{Zadanie 7}
Mamy:\\
\Large$\lambda_k(x)=\prod\limits_{j=0, j \neq k}^n\frac{x-x_j}{x_k-x_j}$\\ \normalsize
a)\\
Wielomian interpolacyjny Lagrange'a ma postać:\\
$L_n(x)=\sum\limits_{k=0}^n f(x_k)\lambda_k(x)$ Weźmy funkcję f(x) stale równą 1. Mamy wtedy:\\
$L_n(x)=\sum\limits_{k=0}^n \lambda_k(x)$. Ale wiemy, że $f(x)$ jest stopnia co najwyżej n, skoro tak, to $L_n(x)=f(x)$, czyli też jest stale równe 1, czyli, że:\\
$L_n(x)=\sum\limits_{k=0}^n \lambda_k(x) \equiv 1$, a to jest to co chcieliśmy pokazać.\\
b)\\
Podobnie jak w poprzednim przypadku, weźmy funkcję $f(x)=x^j$. Gdy $j=0$, $x^0=1$ - punkt 1. W przeciwnym przypadku:\\
$f(x)$ jest stopnia co najwyżej n, więc $L_n(x)=x^j$, a skoro tak to mamy:\\
$L_n(x)=\sum\limits_{k=0}^n x_k^j \lambda_k(x) = x^j$, weźmy teraz $L_n(0)$, mamy:\\
$0^j=0=\sum\limits_{k=0}^n x_k^j \lambda_k(0)$, a to jest to co mieliśmy pokazać. 

\end{document}