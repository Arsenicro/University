\documentclass[a4paper]{article}
\usepackage[left=3cm,right=3cm,top=2cm,bottom=2cm]{geometry} % page settings
\usepackage{enumerate}
\usepackage{hyperref}
\usepackage{graphicx}
\usepackage{amsfonts}
\usepackage{amsthm}
\usepackage{mathtools}
\usepackage{titlesec}
\usepackage{polski}
\usepackage{tikz}
\usepackage[utf8]{inputenc}
\DeclarePairedDelimiter\ceil{\lceil}{\rceil}
\DeclarePairedDelimiter\floor{\lfloor}{\rfloor}
\newcommand{\module}[1]{\left|#1\right|}

\def\checkmark{\tikz\fill[scale=0.3](0,.35) -- (.25,0) -- (1,.7) -- (.25,.15) -- cycle;} 

\titlespacing*{\subsection}
{0ex}{10ex}{3ex}

\title{Lista 4}
\author{Kamil Matuszewski}
\date{4 listopada 2015}

\begin{document}

\maketitle
\setlength{\parindent}{0.5ex}
\setlength{\parskip}{1.5ex}

\begin{center}
\begin{tabular}{|c *{7}{|c} |c|}\hline
1 & 2 & 3 & 4 & 5 & 6 & 7\\
\hline 
\checkmark &\checkmark  &\checkmark  &\checkmark  &\checkmark  &\checkmark  &\checkmark  \\
\hline
\end{tabular}\\
\end{center}

\subsection*{UWAGA}
Nie wstawiam gotowych programów, przedstawię ideę, niech każdy napisze sobie te proste programy w ulubionym języku. Nie wstawiam też analizy wyników, niech każdy się pobawi sam.

\subsection*{Zadanie 1}
\begin{enumerate}[a)]
\item Trywialne
\item $$|b_n-a_n|=|\frac{b_{n-1}-a_{n-1}}{2}|=\ldots=|\frac{b_{n-k}-a_{n-k}}{2^{k}}|=\ldots=|\frac{b_{0}-a_{0}}{2^{n}}| $$
\item Tu będzie trochę machania, można to sformalizować ale jestem leniwy i mi się nie chce, przedstawię ideę.
$$|\epsilon_n|=|\alpha - m_n|$$
Gdzie $\alpha$ to szukany pierwiastek a $m_n$ to środek $n$-tego przedziału. Zastanówmy się, czym może być $m_n$. To może być albo początek, albo koniec $n+1$-ego przedziału. No to rozpatrzmy dwa przypadki. Jeśli to początek przedziału, to znaczy, że $\alpha \geq m_n$. Żeby zmaksymalizować $|\alpha - m_n|$ musimy zmaksymalizować $\alpha$, bo wtedy różnica będzie największa. W takim razie możemy napisać, że $|\alpha - m_n| \leq |b_{n+1}-a_{n+1}|$. Z drugiej strony, jeśli $m_n$ to koniec nowego przedziału, to znaczy, że $\alpha \leq m_n$. Żeby zmaksymalizować różnicę, tym razem musimy zminimalizować $\alpha$, mamy więc, że $|\alpha - m_n| \leq |a_{n+1}-b_{n+1}|=|b_{n+1}-a_{n+1}|$.\\
Skoro tak, to $$|\alpha - m_n| \leq |b_{n+1}-a_{n+1}| = |\frac{b_{0}-a_{0}}{2^{n+1}}|=|2^{-n-1}(b_0-a_0)|$$
Musimy tylko opuścić moduł.\\
Jeśli $a_0>0$ i $b_0>0$ to wiedząc, że $a_0<b_0$ to $b_0-a_0>0$ więc możemy opuścić moduł.\\
Jeśli $a_0<0$ i $b_0>0$ to oczywiście $(b_0-a_0)>0$ więc też możemy opuścić moduł.\\
Jeśli $a_0<0$ i $b_0<0$. $a_0<b_0$ czyli $b_0-a_0>0$ więc też możemy opuścić moduł.\\
Sytuacja, że $a_0>0$ a $b_0<0$ jest niemożliwa.\\
Czyli $2^{-n-1}(b_0-a_0)$ jest zawsze dodatnie, możemy więc opuścić moduł i napisać, że $|\epsilon_n|\leq 2^{-n-1}(b_0-a_0)$
\item Tak, jeśli $\alpha$ jest bardzo blisko $b_0$.
\end{enumerate}

\subsection*{Zadanie 2}
Z zadania 1 wiemy, że $|\epsilon_n|\leq 2^{-n-1}(b_0-a_0)$. Szukamy takiego $n$, żeby $|\epsilon|\geq 2^{-n-1}(b_0-a_0)$
Pomnóżmy stronami przez $\frac{2^{n}}{\epsilon}$. Otrzymujemy:
$$2^{n}\geq \frac{b_0-a_0}{2\epsilon}$$
$$n\geq \log{\frac{b_0-a_0}{2\epsilon}} $$
Zatem, szukanym $n$ jest $\ceil{\log{\frac{b_0-a_0}{2\epsilon}}}$.

\subsection*{Zadanie 3}
Piszemy program który korzystając z zadania $1$ wyliczy nam błąd oszacowania (podpunkt $c$) i porównujemy to z błędem rzeczywistym (czyli środkiem $n$'tego przedziału), wyciągamy wnioski, machamy.

\subsection*{Zadanie 4}
Wyznaczamy przedziały patrząc na wykres naszej funkcji z podpowiedzi (miejsca przecięcia wykresu = miejsca zerowe). Najlepiej, żeby przedziały miały tą samą długość (pewnie jakieś $\pi$). Piszemy metodę bisekcji, używamy zadania $2$ żeby wiedzieć, ile razy ziterować (podstawiamy pod wzór). Odpalamy program i analizujemy jego działanie.

\subsection*{Zadanie 5,6,7,8}
Wszystko robi się tak samo. Piszemy funkcję z miejscem zerowym w naszej szukanej wartości (np. $f(x)=x-\sqrt{a}$). Stosujemy metodę newtona, czyli $x_{k+1}=x_k-\frac{f(x_k)}{f'(x_k)}$ - liczymy pochodną, rozpisujemy wzór. Piszemy jeden program z czterema funkcjami $f(x)$. Porównujemy z wynikiem np. z wolframa i gadamy o tym że fajnie działa.

\end{document}
